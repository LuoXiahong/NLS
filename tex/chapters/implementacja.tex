\section{Implementacja algorytmu w C++}

Prototyp zrealizowany w środowisku Matlab pozwolił na bardzo szybką implementację algorytmu w C++. Miała ona na celu przyspieszenie dokonywanych obliczeń.

Program napisany w C++ ma strukturę jednoplikową i w samym działaniu odbiega od prototypu tylko pod kątem braku wizualizacji wyników. W zamian, rezultat zapisywany jest do pliku tekstowego.

Kompilacja programu została przetestowana pod kompilatorem GCC 5.3.0 na systemie operacyjnym Microsoft Windows 10. Ważną kwestią okazało się dodanie flagi automatycznej optymalizacji \texttt{-Ofast} podczas wykonywania polecenia \texttt{g++ main.cpp}, w przeciwnym wypadku program w C++ wykonywał się znacznie dłużej niż prototyp w Matlabie. 

Podczas uruchamiania programu istnieje możliwość przekazania argumentów z poziomu wiersza poleceń. Parametry te to kolejno nazwa pliku wejściowego, nazwa pliku wyjściowego, parametr $ P $, parametr $ M $ oraz parametr $ \lambda $. W przypadku nie podania argumentów przyjmują one wartości domyślne:

\begin{table}[!htb]
	\centering
	\caption{Domyślne wartości argumentów funkcji \texttt{main}. Parametry liczbowe zostały przyjęte zgodnie z rodziałem \ref{ch:parametry}. }
	\begin{tabular}{|c|l|l|}
		\hline
		Argument & Wartość domyślna \\
		\hline
		Nazwa pliku wejściowego & \texttt{input} \\
		\hline
		Nazwa pliku wyjściowego & \texttt{output} \\
		\hline
		$P$ & $10$ \\
		\hline
		$M$ & $2000$ \\
		\hline
		$\lambda$ & $0.012$ \\
		\hline
	\end{tabular}
	\label{tab:tabela1}
\end{table}

Do przechowywania danych wejściowych oraz sygnału odszumionego wykorzystano darmową bibliotekę Eigen.